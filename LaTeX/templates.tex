%% /* Imagem */
\begin{figure}[H]
    \centering
    \includegraphics[width = 1\linewidth]{img/SECTION/IMAGEM.png}
    \caption{CAPTION}
    \label{fig:LABEL}
\end{figure}

%% /* Múltiplas imagens */
\begin{figure}[ht] 
    \begin{subfigure}[b]{0.5\linewidth}
        \centering
        \includegraphics[width=0.8\linewidth]{img/SECTION/IMAGEM.formato}
        \caption{CAPTION} 
        \label{fig:LABEL-a} 
        \vspace{1ex}
    \end{subfigure}%% 
    \begin{subfigure}[b]{0.5\linewidth}
        \centering
        \includegraphics[width=0.8\linewidth]{img/SECTION/IMAGEM.formato} 
        \caption{CAPTION} 
        \label{fig:LABEL-b} 
        \vspace{1ex}
    \end{subfigure} 
    \begin{subfigure}[b]{0.5\linewidth}
        \centering
        \includegraphics[width=0.8\linewidth]{img/SECTION/IMAGEM.formato}
        \caption{CAPTION} 
        \label{fig:LABEL-c} 
        %%\vspace{4ex}
    \end{subfigure}%% 
    \begin{subfigure}[b]{0.5\linewidth}
        \centering
        \includegraphics[width=0.8\linewidth]{img/SECTION/IMAGEM.formato} 
        \caption{CAPTION} 
        \label{fig:LABEL-d} 
        %%\vspace{4ex}
    \end{subfigure} 
    \caption{CAPTION}
    \label{fig:LABEL}
\end{figure}

%% /* Wrapfigure */
\begin{wrapfigure}[LINHAS]{l}{0.3\textwidth}
    \centering
    \includegraphics[width=0.275\textwidth]{img/SECTION/IMAGEM.formato}
    \caption{CAPTION}
    \label{fig:LABEL}
\end{wrapfigure}

%% /* Imagem e caixa de texto lado a lado */
\begin{tabular}{C{6.5cm}  L{6.5cm}}
    \label{fig:LABEL}
    \raisebox{-2.1\height}{\includegraphics[width = 1\linewidth]{img/SECTION/IMAGEM.formato}}\captionof{figure}{CAPTION} &\
    \noindent\fcolorbox{black}{white}{%
        \minipage[t]{\dimexpr\linewidth-2\fboxsep-2\fboxrule\relax}
            \textbf{À escuta?} $\rightarrow$ Teste 123 experiência
        \endminipage}
\end{tabular} 

%% /* Caixa preta invólucro */
{
\mdfsetup{linewidth=2pt}

\begin{mdframed}
    \noindent Seja $\pmb{\dot{x}} = f(\pmb{x})$, não linear. A equação geral para a linearização de uma função multivariável $f(\pmb{x})$ num ponto $\pmb{p}$ é:
    \vspace{-0.5em}
    $$
        f(\pmb{x}) \approx f(\pmb{p}) + \left.D f\right|_{\pmb{p}} \cdot (\pmb{x} - \pmb{p})
    $$
    
    \noindent onde $\pmb{x}$ é o vetor de variáveis e $\pmb{p}$ o ponto de interesse para a linearização.
\end{mdframed}
}

%% /* Caixa de Teorema/Definição */
\begin{theo}[\underline{Teo/Def.:} <Designação> \cite{}]{teo/def:nome-unico}\label{teo/def:nome-unico}
    >> Enunciar o Teorema/Definição
\end{theo}