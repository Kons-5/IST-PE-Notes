%//==============================--@--==============================//%
\noindent O teste de hipótese visa verificar se uma dada amostra contém ou não informação que deve ser aceite (ou não) no cálculo de parâmetros desconhecidos.

\begin{theo}[\underline{Hipótese paramétrica}]{def:hip-para}\label{def:hip-para}
    \noindent Trata-se de conjetura sobre um parâmetro desconhecido $\theta$, assumindo que se conhece a distribuição da v.a. de interesse $X$ a menos de $\theta$
\end{theo}

\vspace{-1em}
\begin{theo}[\underline{Hipótese nula/alternativa}]{def:null-alter}\label{def:null-alter}
    \noindent Ao confrontarmos dias hipóteses paramétricas:
    \begin{itemize}
        \item a hipótese mais relevante é usualmente designada por hipótese nula, é representada por $H_0$ e correspondente a $H_0\,:\, \theta \in \Theta_0$
        \item $H_1\,:\, \theta \in \Theta_1$ é designada por hipótese alternativa
    \end{itemize}

\noindent Nomeadamente, por exemplo:

$$
    \begin{aligned}
        H_0\,:\, \mu = \mu_0\\
        H_1\,:\, \mu \neq \mu_0
    \end{aligned}
$$

\noindent e neste caso a hipótese alternativa é denominada de \textbf{bilateral}
$$
    \begin{aligned}
        &H_0\,:\, \mu = \mu_0\\
        &H_1\,:\, \mu > \mu_0 \text{ ou }\mu < \mu_0
    \end{aligned}
$$

\noindent e neste caso a hipótese alternativa é denominada de \textbf{unilateral}
\end{theo}

\vspace{-1em}
\begin{theo}[\underline{Erros de 1ª e 2ª espécie} (tipo I e tipo II)]{def:erros}\label{def:erros}
    \noindent As probabilidades de ocorrência de erros da 1ª e 2ª espécie costumam ser representadas por $\alpha$ ou $\beta$ e definem-se do seguinte modo:

    $$
        \begin{aligned}
            &\alpha = P(\text{Erro de 1ª espécie}) = P(\text{Rejeitar }H_0\,|\, H_0)\\
            &\beta = P(\text{Erro de 2ª espécie}) = P(\text{Não rejeitar }H_0\,|\, H_1)
        \end{aligned}
    $$
\end{theo}

\vspace{-1em}
\begin{theo}[\underline{Nível de significância}]{def:significance-level}\label{def:significance-level}
    \noindent O teste de hipóteses deve ser delineado de modo a que se verifique
    $$
        P(\text{Rejeitar }H_0\,|\, H_0) \leq \alpha_0
    $$

    \noindent Onde $a_0$ é designado por nível de significância (congruente com o grau de confiança).
\end{theo}

\noindent \textbf{Região de Rejeição de $\mathbf{H_0}$} (para valores de estatística de teste) --- É habitual escolhermos esta região de modo a que:

\vspace{0.5em}
\begin{itemize}[nolistsep] \small
    \item $W$ satisfaça a condição $P(\text{Rejeitar }H_0\,|\, H_0) = P(T \in W\,|\, H_0) = \alpha_0$ (onde $T$ é designada por estatística de teste).
    \item $W$ seja um intervalo real (ou uma reunião de intervalos reais) que depende de quantis de probabilidade relacionada com $\alpha_0$ e que digam respeito à distribuição exata ou aproximada da estatística de teste sob a validade de $H_0$.
    \item O aspeto de $H_0$ depende da hipótese alternativa $H_1$ e do que significa obter valores inconsistentes com $H_0$.
\end{itemize}

\newpage
\noindent É relevante expor um esqueleto para o procedimento geral de testes de hipótese:

\begin{theo}[\underline{Procedimento geral de testes de hipótese}]{def:significance-level}\label{def:significance-level}
    \noindent Efetuar um teste de hipóteses compreende os sete passos seguintes:
    \begin{enumerate}
        \item \textbf{V.a. de interesse}

        Identificar a v.a. de interesse

        \item \textbf{Situação}

        \noindent Tal como no cálculo dos intervalos de confiança, é necessário adiantar a distribuição da v.a. de interesse, o parâmetro desconhecido que está a ser testado, bem como outros possíveis parâmetros desconhecidos (ou conhecidos), com vista a escolher a variável fulcral apropriada.

        \item \textbf{Hipóteses}

        Enunciar as hipóteses nula e alternativa, de forma análoga à especificada \hyperref[def:null-alter]{acima}.

        \item \textbf{Nível de significância}

        Escolher o nível de significância (tipicamente dado, caso contrário recorrer ao \textit{p-value})

        \item \textbf{Estatística de teste}

        Selecionar a estatatística de teste adequada ($T$) e identificar a sua distribuição (exata ou aproximada) sob $H_0$
        $$
            T \overset{H_0}{\sim} \text{distribuição}
        $$
        Onde T é definida de forma análoga à realizada nos intervalos de confiança, com recurso à variável fulcral adequada (cujo o processo de escolha em nada difere do capítulo anterior).

        \item \textbf{Região de Rejeição}

        Identificar a região de rejeição de $H_0$ para valores da estatística de teste:
        $$
            \left\{\begin{array}{ll}
                 c = F_Z^{-1}(1 - \alpha_0/2) & \text{caso bilateral}\; ]\,-\infty, -c\,[ \cup ]\,c, +\infty\,[\\[4pt]
                 c = F_Z^{-1}(\alpha_0) & \text{caso unilateral esquerdo}\; ]\,-\infty, -c\,[\\[4pt]
                 c = F_Z^{-1}(1 - \alpha_0) & \text{caso unilateral direito}\; ]\,c, \infty\,[
            \end{array}\right.
        $$

        \item \textbf{Decisão}

        Calcular o valor observado da estatística de teste (t) e decidir pela rejeição ou não de $H_0$ ao nível de significância $\alpha_0$.
    \end{enumerate}
\end{theo}

\noindent \textbf{Nota:} Uma qualquer hipótese que seja rejeitado para um dado nível de significância, será rejeitado para qualquer outro superior a este. De forma idêntica, qualquer hipótese que não seja rejeitado para um dado nível de significância, não será rejeitado para qualquer outro inferior a este.
%//==============================--@--==============================//%
\newpage
\subsection[6.1 Cálculo do valor-p]{\hspace*{0.075 em}\raisebox{0.2 em}{$\pmb{\drsh}$} Cálculo do valor-p}

\begin{theo}[\underline{Valor-p}]{def:p-value}\label{def:p-vlue}
    \noindent O valor-p é o maior nível de significância que leva à não rejeição de $H_0$
\end{theo}

\noindent O cálculo do valor-p depende do aspeto da região de $H_0$ (para valores da estatística de teste) e da concretização da estatística de teste, (valor que delimita a área de rejeição), tal como se evidencia abaixo:

\vspace{0.25em}
\begin{figure}[H]
    \centering
    \begin{subfigure}[b]{.315\textwidth}
        \begin{scaletikzpicturetowidth}{\textwidth}
        \begin{tikzpicture}[scale=\tikzscale]
            \begin{axis}[
                domain=-4:4,
                samples=100,
                no markers,
                axis y line = none,
                axis x line = middle,
                xlabel=$t$,
                ylabel=$f(t)$,
                xtick={-1.645,1.645},
                xticklabels={$-|t|$, $|t|$},
                xticklabel style={text height=1.5ex},
                ytick=\empty,
                enlargelimits=false,
                clip=false,
                axis on top,
                grid = none,
                height=6cm,
                width=10cm,
                ]
                \addplot [fill=gray!50, draw=none, domain=-4:-1.645] {exp(-x^2/2)/sqrt(2*pi)} \closedcycle;
                \addplot [fill=gray!50, draw=none, domain=1.645:4] {exp(-x^2/2)/sqrt(2*pi)} \closedcycle;
                \addplot[black, thick] {exp(-x^2/2)/sqrt(2*pi)};
                
                \node at (axis cs:-2.7,0.1) {$\dfrac{\text{valor-p}}{2}$};
                \node at (axis cs:2.7,0.1) {$\dfrac{\text{valor-p}}{2}$};
            \end{axis}
        \end{tikzpicture}
        \end{scaletikzpicturetowidth}%

        %// CAPTION //%
        \caption{%
            \textbf{Hipótese $\mathbf{H_1}$ bilateral} \\
            $P(|T| < |t|) = 2 \times [1 - F_{T | H_0}(|t|)]$
        }
    \end{subfigure}\hfill
    \begin{subfigure}[b]{.315\textwidth}
        \begin{scaletikzpicturetowidth}{\textwidth}
        \begin{tikzpicture}[scale=\tikzscale]
            \begin{axis}[
                domain=-4:4,
                samples=100,
                no markers,
                axis y line = none,
                axis x line = middle,
                xlabel=$t$,
                ylabel=$f(t)$,
                xtick={-1.645,1.645},
                xticklabels={$-|t|$, $|t|$},
                xticklabel style={text height=1.5ex},
                ytick=\empty,
                enlargelimits=false,
                clip=false,
                axis on top,
                grid = none,
                height=6cm,
                width=10cm,
                ]
                \addplot [fill=gray!50, draw=none, domain=-4:-1.645] {exp(-x^2/2)/sqrt(2*pi)} \closedcycle;
                \addplot[black, thick] {exp(-x^2/2)/sqrt(2*pi)};
                
                \node at (axis cs:-2.7,0.1) {$\text{valor-p}$};
            \end{axis}
        \end{tikzpicture}
        \end{scaletikzpicturetowidth}%

        %// CAPTION //%
        \caption{%
            \textbf{Hipótese $\mathbf{H_1}$ unilateral esq.} \\
            $P(T < t\,|\, H_0) = F_{T\,|\, H_0}(t)$
        }
    \end{subfigure}\hfill
    \begin{subfigure}[b]{.315\textwidth}
        \begin{scaletikzpicturetowidth}{\textwidth}
        \begin{tikzpicture}[scale=\tikzscale]
            \begin{axis}[
                domain=-4:4,
                samples=100,
                no markers,
                axis y line = none,
                axis x line = middle,
                xlabel=$t$,
                ylabel=$f(t)$,
                xtick={-1.645,1.645},
                xticklabels={$-|t|$, $|t|$},
                xticklabel style={text height=1.5ex},
                ytick=\empty,
                enlargelimits=false,
                clip=false,
                axis on top,
                grid = none,
                height=6cm,
                width=10cm,
                ]
                \addplot [fill=gray!50, draw=none, domain=1.645:4] {exp(-x^2/2)/sqrt(2*pi)} \closedcycle;
                \addplot[black, thick] {exp(-x^2/2)/sqrt(2*pi)};
                
                \node at (axis cs:2.7,0.1) {$\text{valor-p}$};
            \end{axis}
        \end{tikzpicture}
        \end{scaletikzpicturetowidth}%

        %// CAPTION //%
        \caption{%
            \textbf{Hipótese $\mathbf{H_1}$ unilateral dir.} \\
            $P(T > t\,|\, H_0) = 1 - F_{T\,|\, H_0}(t)$
        }
    \end{subfigure}%
    \label{fig:p-value-find}
\end{figure}

\noindent \textbf{Nota:} Caso a distribuição da v.a. de interesse não seja simétrica, (e.g. qui-quadrado $\chi^2$), e a hipótese alternativa seja bilateral, o valor-p passa a ser definido da seguinte maneira:

$$
    \text{valor-p} = 2 \times \boxed{\min \left\{P(T < t\,|\, H_0)\,,\, P(T > t\,|\, H_0)\right\}}
$$

\begin{theo}[\underline{Teste de ajustamento do qui-quadrado} (hipótese nula simples)]{def:teste-ajusta-qui}\label{def:teste-ajusta-qui}
    \noindent Quando conjeturamos uma única distribuição para a v.a. de interesse $X$, a estatística a utilizar neste teste de ajustamento é

    $$
        T = \sum_{i=1}^{k} \dfrac{(O_i - E_i)}{E_i} \overset{a}{\sim}_{H_0} \mathcal{X}^2_{k - 1}
    $$

    \noindent onde $k$, $O_i$ e $E_i$ representam o número de classes que constam na tabela de frequências, a frequência absoluta observável da classe $i$ e a frequência absoluta esperada sob $H_0$ da classe $i$ respetivamente. Salienta-se que:

    $$
        O_i \sim \text{binomial}(n, p_i)\quad \rightarrow E_i = n P(X \in \text{classe}\,|\, H_0) = n\, p_i^0
    $$
    
    \noindent A região de rejeição da hipótese nula simples $H_0$ é o intervalo à direita onde
    $$
        W = ]c\,,\, +\infty[ \quad \text{onde}\quad c = F^{-1}_{\chi^2_{k - 1}}(1 - \alpha_0)
    $$

    \noindent pois quanto maiores as discrepâncias entre frequências absolutas observadas e esperadas sob $H_0$ das classes, maior é o valor observado da estatística de teste e menos consistente é a hipótese $H_0$ com os dados recolhidos
\end{theo}

%//==============================--@--==============================//%