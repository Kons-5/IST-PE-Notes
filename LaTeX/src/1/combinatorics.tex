%//==============================--@--==============================//%
\clearpage
\subsection[1.3 Notas sobre análise combinatória]{\hspace*{0.075 em}\raisebox{0.2 em}{$\pmb{\drsh}$} Notas sobre análise combinatória}
\label{subsec:combinatorics}

Listamos abaixo alguns resultados respetivos a técnicas de contagem, cruciais no cálculo de probabilidades ao lidarmos com espaços de resultados finitos.

\begin{enumerate}[label=$\bullet$]
    \item \textbf{Permutações de \textit{n} elementos} \hfill $(n!)$
    
    \noindent Número de formas distintas de preenchimento de caixa com $n$ compartimentos, disponde de $n$ elementos e não havendo a possibilidade de repetição no preenchimento. \hfill {\small \textit{Sequências de $n$ elementos distintos}...}
    
    \item \textbf{Arranjos completos\footnotemark[5] de \textit{n} elementos tomados de \textit{x} a \textit{x}} \hfill $(n^x)$

    \noindent Número de formas distintas de preenchimento de caixa com $x$ compartimentos, dispondo de $n$ elementos e havendo a possibilidade de repetição no preenchimento.

    \vspace{-0.425em}
    \hfill {\small \textit{Sequências de $x$ elementos}...}
    
    \item \textbf{Arranjos simples\footnotemark[6] de \textit{n} elementos tomados de \textit{x} a \textit{x}} \hfill $\left(\frac{n!}{(n-x)!}\right)$

    \noindent Número de formas distintas de preenchimento de caixa com $x$ compartimentos, dispondo de $n$ elementos e não havendo possibilidade de repetição no preenchimento.

    \vspace{-0.425em}
    \hfill {\small \textit{Sequências de $x$ elementos distintos}...}
    
    \item \textbf{Combinações de \textit{n} elementos tomados de \textit{x} a \textit{x}} \hfill $\left(\binom{n}{x} = \frac{n!}{x!(n-x)!}\right)$

    \noindent Número de conjuntos de cardinal $x$ (logo com elementos distintos) que podem ser formados com $n$ elementos. \hfill {\small \textit{Conjuntos de $x$ elementos distintos}...}
    
    \item \textbf{Permutações de \textit{n} elementos de \textit{k} tipos distintos} \hfill $\left(\frac{n!}{n_1!n_2!\dots n_k!}\right)$

    \noindent Número de formas distintas de preenchimento de caixa com $n$ compartimentos, dispondo de $n$ elementos de $k$ tipos distintos, onde $n_i$ representa o número de elementos de tipo $i=1,2,\dots,k$, e $\sum_{i=1}^{k} n_i = n$. 
    
    \vspace{-0.425em}
    \hfill {\small \textit{Sequências de $n$ elementos de $k$ tipos}...} 
    
    \item \textbf{Binómio de Newton}
    $$
        (a+b)^n = \sum_{x=0}^{n} \binom{n}{x} a^x b^{n-x}
    $$

    \begin{figure}[H]
    \centering
        \tightbox{
            \begin{tabular}{>{$n=}l<{$\hspace{12pt}}*{13}{c}}
                0 &&&&&&&1&&&&&&\\
                1 &&&&&&1&&1&&&&&\\
                2 &&&&&1&&2&&1&&&&\\
                3 &&&&1&&3&&3&&1&&&\\
                4 &&&1&&4&&6&&4&&1&&\\
                5 &&1&&5&&10&&10&&5&&1&\\
                6 &1&&6&&15&&20&&15&&6&&1
            \end{tabular}
            $$
                \qquad\qquad\;\:\pmb{\dots}
            $$
        }
        \caption{Triângulo de Pascal}
    \end{figure}
\end{enumerate}

\footnotetext[5]{%
    Ou arranjos \underline{\textit{com}} repetições.
}
\footnotetext[6]{%
    Ou arranjos \underline{\textit{sem}} repetições.
}

%//==============================--@--==============================//%