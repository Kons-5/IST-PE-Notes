%//==============================--@--==============================//%
\subsection[5.1 IC para o valor esperado e variância conhecida]{\hspace*{0.075 em}\raisebox{0.2 em}{$\pmb{\drsh}$} IC para o valor esperado e variância conhecida}

\noindent Supondo inicialmente que a v.a. de interesse possui \textbf{distribuição normal}, temos:

\begin{mdframed}
    \begin{itemize}[leftmargin=*]
        \item \textbf{Situação}
        $$
            X \sim \text{normal}(\mu, \sigma^2) \;\rightarrow\; \left\{
            \begin{aligned}
                &\pmb{\mu} \mkern9.5mu\texttt{-}\mkern9.5mu \texttt{\color{red} DESCONHECIDO} \\
                &\sigma^2 \mkern3mu\texttt{-}\mkern9mu \texttt{\color{green} CONHECIDO}
            \end{aligned}\right.
        $$
        
        \item \textbf{Passo 1 --- Seleção da v.a. fulcral para $\mu$}
        $$
            Z = \dfrac{\overline{X} - \mu}{\sigma/\sqrt{n}} \sim \text{normal}(0,1)
        $$
    
        \item \textbf{Passo 2 --- Obtenção dos quantis de probabilidade}
        $$
            (a_\alpha, b_\alpha)\, :\, \left\{\begin{array}{ll}
                 a_\alpha = \Phi(\alpha/2) = -\Phi(1 - \alpha/2)\\
                 b_\alpha = \Phi(1 - \alpha/2)
            \end{array}\right.
        $$
    
        \item \textbf{Passo 3 --- Inversão de igualdades}
        $$
            \begin{aligned}
                &P(a_\alpha \leq Z \leq b_\alpha) = 1 - \alpha\\
                &P\left[a_\alpha \leq \dfrac{\overline{X} - \mu}{\sigma/\sqrt{n}} \leq b_\alpha\right] = 1 - \alpha\\
                &P\left[\overline{X} - b_\alpha\cdot \dfrac{\sigma}{\sqrt{n}} \leq \mu \leq \overline{X} + a_\alpha\cdot \dfrac{\sigma}{\sqrt{n}}\right] = 1 - \alpha\\
            \end{aligned}
        $$
    
        \item\textbf{Passo 4 --- Concretização}
        $$
            IC_{(1 - \alpha)\cdot 100\%} = \left[\overline{x} \pm \Phi(1 - \alpha/2) \cdot \dfrac{\sigma}{\sqrt{n}}\right]
        $$
    \end{itemize}
\end{mdframed}

\newpage
\noindent Caso a população possua \textbf{distribuição arbitrária} (e uma amostra suficientemente grande, respetivamente $\ge 30$) o procedimento é idêntico ao enunciado acima, com exceção da v.a. fulcral e respetivo IC, que passam agora a ser denominados por \textit{aproximados} (consequência direta da utilização do  \hyperref[def:tlc]{TLC}).

\begin{itemize}[leftmargin=*,noitemsep]
    \item \textbf{Passo 1 --- Seleção da v.a. fulcral para $\mu$}
    $$
        Z = \dfrac{\overline{X} - \mu}{\sigma/\sqrt{n}} \overset{a}{\sim} \text{normal}(0,1)
    $$
    \item\textbf{Passo 4 --- Concretização}
    $$
        IC_{(1 - \alpha)\cdot 100\%} \simeq \left[\overline{x} \pm \Phi(1 - \alpha/2) \cdot \dfrac{\sigma}{\sqrt{n}}\right]
    $$
\end{itemize}

\subsection[5.2 IC para o valor esperado e variância desconhecida]{\hspace*{0.075 em}\raisebox{0.2 em}{$\pmb{\drsh}$} IC para o valor esperado e variância desconhecida}

\noindent Supondo inicialmente que a v.a. de interesse possui \textbf{distribuição normal}, temos:

\begin{mdframed}
    \begin{itemize}[leftmargin=*,noitemsep]
        \item \textbf{Situação}
        $$
            X \sim \text{normal}(\mu, \sigma^2) \;\rightarrow\; \left\{
            \begin{aligned}
                &\pmb{\mu} \mkern9.5mu\texttt{-}\mkern9.5mu \texttt{\color{red} DESCONHECIDO} \\
                &\sigma^2 \mkern3mu\texttt{-}\mkern9mu \texttt{\color{red} DESCONHECIDA}
            \end{aligned}\right.
        $$
        
        \item \textbf{Passo 1 --- Seleção da v.a. fulcral para $\mu$}
        $$
            Z = \dfrac{\overline{X} - \mu}{S/\sqrt{n}} \sim t_{n - 1}\qquad
            S^2 = \dfrac{1}{n - 1} \sum_{i = 1}^{n}(X_i - \overline{X})^2
        $$
    
        \item \textbf{Passo 2 --- Obtenção dos quantis de probabilidade}
         $$
            (a_\alpha, b_\alpha)\, :\, \left\{\begin{array}{ll}
                 a_\alpha = F_t(\alpha/2) = -F_t(1 - \alpha/2)\\
                 b_\alpha = F_t(1 - \alpha/2)
            \end{array}\right.
        $$
    
        \item \textbf{Passo 3 --- Inversão de igualdades}
        $$
            \begin{aligned}
                &P(a_\alpha \leq Z \leq b_\alpha) = 1 - \alpha\\
                &P\left[a_\alpha \leq \dfrac{\overline{X} - \mu}{\sigma/\sqrt{n}} \leq b_\alpha\right] = 1 - \alpha\\
                &P\left[\overline{X} - b_\alpha\cdot \dfrac{S}{\sqrt{n}} \leq \mu \leq \overline{X} + a_\alpha\cdot \dfrac{S}{\sqrt{n}}\right] = 1 - \alpha\\
            \end{aligned}
        $$
    
        \item\textbf{Passo 4 --- Concretização}
        $$
            IC_{(1 - \alpha)\cdot 100\%} = \left[\overline{x} \pm F_t(1 - \alpha/2) \cdot \dfrac{s}{\sqrt{n}}\right]
        $$
    \end{itemize}
\end{mdframed}

\noindent Caso a população possua \textbf{distribuição arbitrária} (e uma amostra suficientemente grande, respetivamente $\ge 30$) o procedimento é idêntico ao enunciado acima, com exceção da v.a. fulcral, que passa agora a ter distribuição aproximadamente normal, ao invés de \textit{t-student} (consequência direta da utilização do  \hyperref[def:tlc]{TLC}). \hfill $\implies Z = \frac{\overline{X} - \mu}{S/\sqrt{n}} \overset{a}{\sim} \text{normal}(0,1)$

\subsection[5.3 IC para a variância e valor esperado desconhecido]{\hspace*{0.075 em}\raisebox{0.2 em}{$\pmb{\drsh}$} IC para a variância e valor esperado desconhecido}

\noindent Supondo inicialmente que a v.a. de interesse possui \textbf{distribuição normal}, (caso a distribuição da v.a. de interessa não seja dada, admitimos sempre distribuição normal) temos:

\begin{mdframed}
    \begin{itemize}[leftmargin=*]
        \item \textbf{Situação}
        $$
            X \sim \text{normal}(\mu, \sigma^2) \;\rightarrow\; \left\{
            \begin{aligned}
                &\mu \mkern9.5mu\texttt{-}\mkern9.5mu \texttt{\color{red} DESCONHECIDO}\\
                &\pmb{\sigma^2} \mkern3mu\texttt{-}\mkern9mu \texttt{\color{red} DESCONHECIDA}
            \end{aligned}\right.
        $$
    
        \item \textbf{Passo 1 --- Seleção da v.a. fulcral para $\mu$}
        $$
            Z = \dfrac{S^2 (n - 1)}{\sigma} \sim \chi^2_{n - 1}\qquad
            S^2 = \dfrac{1}{n - 1} \sum_{i = 1}^{n}(X_i - \overline{X})^2
        $$
    
        \item \textbf{Passo 2 --- Obtenção dos quantis de probabilidade}
         $$
            (a_\alpha, b_\alpha)\, :\, \left\{\begin{array}{ll}
                 a_\alpha = F_{\chi^2}(\alpha/2) \\
                 b_\alpha = F_{\chi^2}(1 - \alpha/2)
            \end{array}\right.
        $$
        \noindent Já que a distribuição \textit{qui-quadrado} possui assimetria positiva.
    
        \vspace{1 em}
        \item \textbf{Passo 3 --- Inversão de igualdades}
    
        $$
            \begin{aligned}
                &P(a_\alpha \leq Z \leq b_\alpha) = 1 - \alpha\\
                &P\left[a_\alpha \leq \dfrac{S^2 (n - 1)}{\sigma} \leq b_\alpha\right] = 1 - \alpha\\
                &P\left[\dfrac{(n-1)S^2}{b_\alpha} \leq \sigma^2 \leq \dfrac{(n-1)S^2}{b_\alpha}\right] = 1 - \alpha\\
            \end{aligned}
        $$
    
        \item\textbf{Passo 4 --- Concretização}
    
        $$
            IC_{(1 - \alpha)\cdot 100\%} = \left[\dfrac{(n-1)S^2}{F_{\chi^2}(1 - \alpha/2)}\,;\, \dfrac{(n-1)S^2}{F_{\chi^2}(\alpha/2)}\right]
        $$
    \end{itemize}
\end{mdframed}

\newpage
\subsection[5.4 IC's para uma probabilidade de sucesso e outros parâmetros de população não normais uniparamétricas]{\hspace*{0.075 em}\raisebox{0.2 em}{$\pmb{\drsh}$} IC's para uma probabilidade de sucesso e outros parâmetros de população não normais uniparamétricas}

\noindent Importa adiantar IC aproximados para a probabilidade de sucesso de uma prova de Bernoulli, bem como outros parâmetros de populações não normais uniparamétricas. Os passos de cálculo para o intervalo de tempo em nada se alteram relativamente aos realizados nas secções anteriores. Neste sentido, é relevante evidenciar apenas as variáveis fulcrais interatuantes no cálculo do IC:

\subsubsection[5.4.1 Probabilidade de sucesso num ensaio de Bernoulli]{$\pmb{\rightarrow}$ Probabilidade de sucesso num ensaio de Bernoulli}

$$
    \boxed{Z = \dfrac{\overline{X} - p}{\sqrt{\overline{X}(1-\overline{X})/n}} \overset{a}{\sim} \text{normal}(0,1)}
$$

\noindent Desde que o tamanho da amostra justifique o uso do teorema do limite central. Note-se ainda que $E(\overline{X}) = E(X) = p$, $V(\overline{X}) = V(X)/n = p(1 - p)/n \simeq \overline{X}(1-\overline{X})/n$ de acordo com a distribuição binomial.

\subsubsection[5.4.2 Valor médio de uma população Poisson]{$\pmb{\rightarrow}$ Valor médio de uma população Poisson}

$$
    \boxed{Z = \dfrac{\overline{X} - \lambda}{\sqrt{\overline{X}/n}} \overset{a}{\sim} \text{normal}(0,1)}
$$

\noindent Novamente, Desde que o tamanho da amostra justifique o uso do teorema do limite central, onde $V(\overline{X}) = \lambda/n \simeq \overline{X}/n$.

\subsubsection[5.4.3 Rate de uma população exponencial]{$\pmb{\rightarrow}$ Rate de uma população de exponencial}

\noindent Recorrendo ao Teorema do Limite Central:
$$
    \boxed{%
    \begin{aligned}
        Z &= \dfrac{\overline{X} - \lambda^{-1}}{\lambda^{-1}/\sqrt{n}}\\[4pt]
        &= \Aboxed{(\lambda \overline{X} - 1)\cdot \sqrt{n} \overset{a}{\sim} \text{normal}(0,1)}
    \end{aligned}
    }
$$

%//==============================--@--==============================//%